
\section{Hypothesis, Objectives and Scope}
\label{cap:1-Hypothesis_Objectives_and_Scope}

Based on the reviewed barriers, pitfalls and limitations of the current State of the Art in IoT approaches and their implications in intelligent work environments, the hypothesis of the present dissertation is:


\textbf{Hypothesis:} Incorporating users into the decision-making loop of AI for task allocation in CS projects through the Hybrid intelligence (HI) paradigm increases the acceptance and \textcolor{red}{promote the} adoption of crowdsourcing sensing technology, effectively closing the Human-AI feedback loop. 


Hence, this dissertation sets the following goal in order to validate the aforementioned hypothesis:

\textbf{Goal:} To design and implement a Hybrid Intelligence-based system for task allocation in Citizen Science projects, which integrates the user into the decision-making loop of AI, increasing their acceptance and \textcolor{red}{promote the} adoption of crowdsourcing sensing technology.

This general goal can be achieved by addressing the following objectives:
\begin{enumerate}
  \item To study the current state of the art on Human-AI collaboration 
  \item .
  \item Define a customized problem statement to be able to integrate the users in the task allocation problems in the context of Citizen Science projects.
  \item To design and implement a Hybrid Intelligence-based system for task allocation in Citizen Science projects.
  \item 
\end{enumerate}

The resulting optimized and interactive system should also fulfill the following requirements:
\begin{enumerate}
  \item .
\end{enumerate}

Beyond that, the work presented in this dissertation does not deal with the following conditions:
\begin{enumerate}
  \item .
\end{enumerate}


